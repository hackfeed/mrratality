\chapter{Аналитическая часть}

В данном разделе описана метрика MRR, постановка задачи построения MRR отчета. Представлен анализ способов хранения данных и систем управления базами данных, оптимальных для решения поставленной задачи. Приведен анализ существующих решений.

\section{Формализация задачи}

Принятие решений на основе данных является обязательным условием при развитии бизнеса. Однако подход основанный на данных сильно зависит от самих данных, правильности их сбора и обработки. Зачастую, бизнес не умеет собирать нужные данные правильно, также он сталкивается с проблемами их обработки.

Ключевые данные, необходимые для принятия решений, это данные о продажах, а именно, кто, когда, сколько платит. При этом, в современном мире данные о платежах обычно имеются в любом бизнесе.

В качестве примера будут рассматриваться приложения с бизнес-моделью подписки, которые имеют отличительную особенность -- рекуррентные платежи. Вести учет движения средств в такой модели, используя стандартный бухгалтерский подход, неправильно, так как он не учитывает будущие поступления, кроме того бухгалтерский подход нормирует все поступления на сутки, а следовательно при оплате клиентом месячной подписки, например, с 20 января по 19 февраля получится вклад от услуги в оба месяца, при этом вклад в каждый месяц будет рассчитан пропорционально числу дней в месяце. При этом клиент понимает, что подписка была оформлена в январе, а следующий платеж будет совершен в феврале. Для учета таких платежей придумали отчет MRR, в котором платежи нормируются на месяц начала действия оплаченного периода, таким образом, оплаченный период с 20 января по 19 февраля целиком будет относиться к январю.

\section{Метрика MRR}

Monthly Recurring Revenue -- метрика для принятия решений, отображающая регулярную месячную выручку. Применяется в SaaS-приложениях \cite{saas} с бизнес-моделью подписок \cite{mrr}.

\subsection{Подготовка данных для MRR отчета}

Для того, чтобы построить MRR отчет, в исходных данных обязательно должна быть следующая информация:

\begin{itemize}
	\item \texttt{customer\_id} -- идентификатор клиента, уникальное значение, определяющее каждого отдельного клиента;
	\item \texttt{paid\_amount} -- сумма, заплаченная клиентом;
	\item \texttt{paid\_plan} -- тип тарифного плана (месячный, трехмесячный, полугодовой);
	\item \texttt{period\_start} -- дата выставления счета (начало периода);
\end{itemize}

В таблице \ref{tab:mrr1} приведен пример начальной выгрузки данных.

\begin{table}[H]
	\centering
	\caption{Начальная выгрузка данных}
	\label{tab:mrr1}
	\resizebox{\textwidth}{!}{%
		\begin{tabular}{|c|c|c|c|}
			\hline
			\textbf{customer\_id} & \textbf{period\_start} & \textbf{paid\_plan} & \textbf{paid\_amount} \\ \hline
			1                     & 01.06.2019             & annually            & 90                    \\ \hline
			2                     & 01.06.2019             & annually            & 90                    \\ \hline
			3                     & 01.06.2019             & monthly             & 10                    \\ \hline
			3                     & 01.08.2019            & monthly             & 10                    \\ \hline
			4                     & 01.06.2019             & monthly             & 10                    \\ \hline
			5                     & 01.06.2019             & monthly             & 10                    \\ \hline
			6                     & 01.06.2019             & monthly             & 10                    \\ \hline
			7                     & 01.06.2019             & annually            & 90                    \\ \hline
			8                     & 01.06.2019             & monthly             & 10                    \\ \hline
			9                     & 01.06.2019             & monthly             & 10                    \\ \hline
			10                    & 01.06.2019             & monthly             & 10                    \\ \hline
			11                    & 01.06.2019             & monthly             & 10                    \\ \hline
			12                    & 01.06.2019             & annually            & 90                    \\ \hline
			13                    & 01.06.2019             & annually            & 90                    \\ \hline
			14                    & 01.06.2019             & monthly             & 10                    \\ \hline
		\end{tabular}%
	}
\end{table}

Для построения отчета необходимо вычислить дату окончания оплаченного периода. Для этого необходимо правильно учитывать количество дней в месяцах, фактор високосного года и т.п. 

В таблице \ref{tab:mrr2} приведен пример доработанной выгрузки данных.

\begin{table}[H]
	\centering
	\caption{Доработанная выгрузка данных}
	\label{tab:mrr2}
	\resizebox{\textwidth}{!}{%
		\begin{tabular}{|c|c|c|c|c|}
			\hline
			\textbf{customer\_id} & \textbf{period\_start} & \textbf{paid\_plan} & \textbf{paid\_amount} & \textbf{period\_end} \\ \hline
			1                     & 01.06.2019             & annually            & 90                    & 31.05.2020           \\ \hline
			2                     & 01.06.2019             & annually            & 90                    & 31.05.2020           \\ \hline
			3                     & 01.06.2019             & monthly             & 10                    & 30.06.2019           \\ \hline
			3                     & 01.08.2019             & monthly             & 10                    & 31.08.2019           \\ \hline
			4                     & 01.06.2019             & monthly             & 10                    & 30.06.2019           \\ \hline
			5                     & 01.06.2019             & monthly             & 10                    & 30.06.2019           \\ \hline
			6                     & 01.06.2019             & monthly             & 10                    & 30.06.2019           \\ \hline
			7                     & 01.06.2019             & annually            & 90                    & 31.05.2020           \\ \hline
			8                     & 01.06.2019             & monthly             & 10                    & 30.06.2019           \\ \hline
			9                     & 01.06.2019             & monthly             & 10                    & 30.06.2019           \\ \hline
			10                    & 01.06.2019             & monthly             & 10                    & 30.06.2019           \\ \hline
			11                    & 01.06.2019             & monthly             & 10                    & 30.06.2019           \\ \hline
			12                    & 01.06.2019             & annually            & 90                    & 31.05.2020           \\ \hline
			13                    & 01.06.2019             & annually            & 90                    & 31.05.2020           \\ \hline
			14                    & 01.06.2019             & monthly             & 10                    & 30.06.2019           \\ \hline
		\end{tabular}%
	}
\end{table}

После вычисления даты окончания периода необходимо создать промежуточный набор данных, в котором будет учтено распределение уплаченных средств согласно периодам (нормированным на начало месяца) в которые они попадают. Фактически этот набор данных может представлять из себя таблицу, в которой строки -- идентификаторы клиентов, столбцы -- периоды, в ячейках которых указывается сумма, уплаченная клиентом за этот период. Сумма вычисляется в зависимости от того, какой тарифный план использует клиент. 

В таблице \ref{tab:mpp1} приведен пример таблицы распределения уплаченных средств согласно периодам.

\begin{table}[H]
	\centering
	\caption{Распределение уплаченных средств согласно периодам}
	\label{tab:mpp1}
	\resizebox{\textwidth}{!}{%
	\begin{tabular}{|c|c|c|c|c|c|c|c|c|c|c|c|c|}
		\hline
		\textbf{} & 06.2019 & 07.2019 & 08.2019 & 09.2019 & 10.2019 & 11.2019 & 12.2019 & 01.2020 & 02.2020 & 03.2020 & 04.2020 & 05.2020 \\ \hline
		1         & 7.5     & 7.5     & 7.5     & 7.5     & 7.5     & 7.5     & 7.5     & 7.5     & 7.5     & 7.5     & 7.5     & 7.5     \\ \hline
		2         & 7.5     & 7.5     & 7.5     & 7.5     & 7.5     & 7.5     & 7.5     & 7.5     & 7.5     & 7.5     & 7.5     & 7.5     \\ \hline
		3         & 10      & 0       & 0       & 0       & 0       & 0       & 0       & 0       & 0       & 0       & 0       & 0       \\ \hline
		3         & 0       & 0       & 10      & 0       & 0       & 0       & 0       & 0       & 0       & 0       & 0       & 0       \\ \hline
		4         & 10      & 0       & 0       & 0       & 0       & 0       & 0       & 0       & 0       & 0       & 0       & 0       \\ \hline
		5         & 10      & 0       & 0       & 0       & 0       & 0       & 0       & 0       & 0       & 0       & 0       & 0       \\ \hline
		6         & 10      & 0       & 0       & 0       & 0       & 0       & 0       & 0       & 0       & 0       & 0       & 0       \\ \hline
		7         & 7.5     & 7.5     & 7.5     & 7.5     & 7.5     & 7.5     & 7.5     & 7.5     & 7.5     & 7.5     & 7.5     & 7.5     \\ \hline
		8         & 10      & 0       & 0       & 0       & 0       & 0       & 0       & 0       & 0       & 0       & 0       & 0       \\ \hline
		9         & 10      & 0       & 0       & 0       & 0       & 0       & 0       & 0       & 0       & 0       & 0       & 0       \\ \hline
		10        & 10      & 0       & 0       & 0       & 0       & 0       & 0       & 0       & 0       & 0       & 0       & 0       \\ \hline
		11        & 10      & 0       & 0       & 0       & 0       & 0       & 0       & 0       & 0       & 0       & 0       & 0       \\ \hline
		12        & 7.5     & 7.5     & 7.5     & 7.5     & 7.5     & 7.5     & 7.5     & 7.5     & 7.5     & 7.5     & 7.5     & 7.5     \\ \hline
		13        & 7.5     & 7.5     & 7.5     & 7.5     & 7.5     & 7.5     & 7.5     & 7.5     & 7.5     & 7.5     & 7.5     & 7.5     \\ \hline
		14        & 10      & 0       & 0       & 0       & 0       & 0       & 0       & 0       & 0       & 0       & 0       & 0       \\ \hline
	\end{tabular}%
	}
\end{table}

Важно понимать, что в полученном наборе данных могут быть клиенты, которые платили несколько раз (например, месячная подписка с разницей в месяц). Такие клиенты будут иметь несколько строчек в таблице. В таблице \ref{tab:mpp1} есть такой случай -- это клиент с идентификатором \texttt{3}.

Теперь необходимо свести этот отчет к каждому уникальному клиенту, чтобы считать динамику MRR (\texttt{new}, \texttt{old}, \texttt{reactivation}, \texttt{expansion}, \texttt{contraction}, \texttt{churn}. Подробнее о каждом критерии будет сказано далее).

В таблице \ref{tab:mpp2} приведен пример нормированной по клиентам таблицы распределения уплаченных средств согласно периодам.

\begin{table}[H]
	\centering
	\caption{Нормированное распределение уплаченных средств согласно периодам}
	\label{tab:mpp2}
	\resizebox{\textwidth}{!}{%
		\begin{tabular}{|c|c|c|c|c|c|c|c|c|c|c|c|c|}
			\hline
			\textbf{} & 06.2019 & 07.2019 & 08.2019 & 09.2019 & 10.2019 & 11.2019 & 12.2019 & 01.2020 & 02.2020 & 03.2020 & 04.2020 & 05.2020 \\ \hline
			1         & 7.5     & 7.5     & 7.5     & 7.5     & 7.5     & 7.5     & 7.5     & 7.5     & 7.5     & 7.5     & 7.5     & 7.5     \\ \hline
			2         & 7.5     & 7.5     & 7.5     & 7.5     & 7.5     & 7.5     & 7.5     & 7.5     & 7.5     & 7.5     & 7.5     & 7.5     \\ \hline
			3         & 10      & 0       & 10      & 0       & 0       & 0       & 0       & 0       & 0       & 0       & 0       & 0       \\ \hline
			4         & 10      & 0       & 0       & 0       & 0       & 0       & 0       & 0       & 0       & 0       & 0       & 0       \\ \hline
			5         & 10      & 0       & 0       & 0       & 0       & 0       & 0       & 0       & 0       & 0       & 0       & 0       \\ \hline
			6         & 10      & 0       & 0       & 0       & 0       & 0       & 0       & 0       & 0       & 0       & 0       & 0       \\ \hline
			7         & 7.5     & 7.5     & 7.5     & 7.5     & 7.5     & 7.5     & 7.5     & 7.5     & 7.5     & 7.5     & 7.5     & 7.5     \\ \hline
			8         & 10      & 0       & 0       & 0       & 0       & 0       & 0       & 0       & 0       & 0       & 0       & 0       \\ \hline
			9         & 10      & 0       & 0       & 0       & 0       & 0       & 0       & 0       & 0       & 0       & 0       & 0       \\ \hline
			10        & 10      & 0       & 0       & 0       & 0       & 0       & 0       & 0       & 0       & 0       & 0       & 0       \\ \hline
			11        & 10      & 0       & 0       & 0       & 0       & 0       & 0       & 0       & 0       & 0       & 0       & 0       \\ \hline
			12        & 7.5     & 7.5     & 7.5     & 7.5     & 7.5     & 7.5     & 7.5     & 7.5     & 7.5     & 7.5     & 7.5     & 7.5     \\ \hline
			13        & 7.5     & 7.5     & 7.5     & 7.5     & 7.5     & 7.5     & 7.5     & 7.5     & 7.5     & 7.5     & 7.5     & 7.5     \\ \hline
			14        & 10      & 0       & 0       & 0       & 0       & 0       & 0       & 0       & 0       & 0       & 0       & 0       \\ \hline
		\end{tabular}%
	}
\end{table}

Каждая строка полученной таблицы -- это клиент, а каждый столбец -- сумма поступлений от клиента в этот период.

\subsection{Построение отчета о динамике MRR}

Для того, чтобы считать динамику MRR, необходимо ввести компоненты метрики:

\begin{itemize}
	\item \texttt{New} -- поступления от клиента, который еще ни разу не оплачивал подписку в указанный промежуток. Если ориентироваться по таблице \ref{tab:mpp2}, то это значит, что левее выбранного периода поступлений нет;
	\item \texttt{Old} -- поступления от клиента в выбранном периоде равны поступлениям в предыдущем периоде;
	\item \texttt{Reactivation} -- поступления от клиента в выбранном периоде не равны нулю, при этом в прошлом периоде они были равны нулю. Также были поступления до предыдущего периода;
	\item \texttt{Expansion} -- поступления от клиента в выбранном периоде больше поступлений в предыдущем периоде, причем поступления в предыдущем периоде не равны нулю;
	\item \texttt{Contraction} -- поступления от клиента в выбранном периоде меньше поступлений в предыдущем периоде;
	\item \texttt{Churn} -- поступления от клиента в выбранном периоде равны нулю, а в предыдущем периоде больше нуля.
\end{itemize}

Итоговый MRR за выбранный период можно расчитать по формуле \ref{eq:mrr}:
\begin{equation}
	\label{eq:mrr}
	MRR_i = New_i + Old_i + Reactivation_i + Expansion_i - Contraction_i - Churn_i
\end{equation}

Произведя вычисления по каждому периоду, можно получить результат, представленный в таблице \ref{tab:mrrcomp}.

\begin{table}[H]
	\centering
	\caption{Результат вычисления MRR}
	\label{tab:mrrcomp}
	\resizebox{\textwidth}{!}{%
		\begin{tabular}{|c|c|c|c|c|c|c|c|c|c|c|c|c|}
			\hline
			\textbf{}    & 06.2019 & 07.2019 & 08.2019 & 09.2019 & 10.2019 & 11.2019 & 12.2019 & 01.2020 & 02.2020 & 03.2020 & 04.2020 & 05.2020 \\ \hline
			New          & 127.5   & 0       & 0       & 0       & 0       & 0       & 0       & 0       & 0       & 0       & 0       & 0       \\ \hline
			Old          & 0       & 37.5    & 37.5    & 37.5    & 37.5    & 37.5    & 37.5    & 37.5    & 37.5    & 37.5    & 37.5    & 37.5    \\ \hline
			Expansion    & 0       & 0       & 0       & 0       & 0       & 0       & 0       & 0       & 0       & 0       & 0       & 0       \\ \hline
			Reactivation & 0       & 0       & 10      & 0       & 0       & 0       & 0       & 0       & 0       & 0       & 0       & 0       \\ \hline
			Contraction  & 0       & 0       & 0       & 0       & 0       & 0       & 0       & 0       & 0       & 0       & 0       & 0       \\ \hline
			Churn        & 0       & -90     & 0       & -10     & 0       & 0       & 0       & 0       & 0       & 0       & 0       & 0       \\ \hline
			MRR          & 127.5   & -52.5   & 47.5    & 27.5    & 37.5    & 37.5    & 37.5    & 37.5    & 37.5    & 37.5    & 37.5    & 37.5    \\ \hline
		\end{tabular}%
	}
\end{table}

\section{Базы данных и системы управления базами данных}

В задаче построения MRR отчета важную роль играет выбор модели хранения данных, которая будет использоваться для хранения данных о пользователях и анализируемых транзакций.

Для персистентного хранения данных используются базы данных \cite{dbms}. Для управления базами данных используются системы управления базами данных (сокращенно СУБД) \cite{dbms}. Система управления базами данных -- совокупность программных и лингвистических средств общего или специального назначения, обеспечивающих управление созданием и использованием баз данных.

\section{Хранение данных о пользователе}

Система, разрабатываемая в рамках курсового проекта, предполагает собой приложение, доступное всем желающим. Поэтому необходимо предусмотреть наличие нескольких пользователей, каждый из которых будет иметь возможность загружать свои наборы данных для аналитики.

\subsection{Классификация баз данных по способу обработки}

По способу обработки базы данных делятся на две группы -- реляционные и нереляционные базы данных. Каждый из двух типов служит для выполнения определенного рода задач.

\subsubsection{Реляционные базы данных (SQL)}

Данные реляционных баз хранятся в виде таблиц и строк, таблицы могут иметь связи с другими таблицами через внешние ключи, таким образом образуя некие отношения.

Реляционные базы данных используют язык SQL \cite{sql}. Структура таких баз данных позволяет связывать информацию из разных таблиц с помощью внешних ключей (или индексов), которые используются для уникальной идентификации любого атомарного фрагмента данных в этой таблице. Другие таблицы могут ссылаться на этот внешний ключ, чтобы создать связь между частями данных и частью, на которую указывает внешний ключ.

SQL используют универсальный язык структурированных запросов для определения и обработки данных. Это накладывает определенные ограничения: прежде чем начать обработку, данные надо разместить внутри таблиц и описать.

\subsubsection{Нереляционные базы данных (NoSQL)}

Данные нереляционных баз данных не имеют общего формата. Они могут представляться в виде документов (Mongo \cite{mongo}, Tarantool \cite{tarantool}), пар ключ-значение (Redis \cite{redis}), графовых представляниях (Neo4j \cite{neo4j}).

Динамические схемы для неструктурированных данных позволяют:

\begin{itemize}
	\item ориентировать информацию на столбцы или документы;
	\item основывать ее на графике;
	\item организовывать в виде хранилища KeyValue;
	\item создавать документы без предварительного определения их структуры, использовать разный синтаксис;
	\item добавлять поля непосредственно в процессе обработки.
\end{itemize}

\subsection{Выбор модели хранения данных для решения задачи}

Для решения задачи NoSQL базы данных выглядят лучше в контексте работы по нескольким причинам:

\begin{itemize}
	\item задача не предполагает наличие отношений. В системе будет присутствовать одна коллекция пользователей;
	\item задача предполагает отслеживание наборов данных (ссылок на наборы), которыми располагает пользователь , причем этот набор постоянно изменяется по размеру, следовательно присутствует элемент динамичности схемы. Следует рассматривать документоориентированные СУБД.
\end{itemize}

\subsection{Обзор NoSQL СУБД}

Существует большое количество NoSQL СУБД. В данном подразделе будут рассмотрены популярные некоммерческие NoSQL документоориентированные СУБД, которые могут быть использованы в проекте.

\subsubsection{Couchbase}

Couchbase (Couchbase Server) \cite{couchbase} -- система управления базами данных класса NoSQL, предоставляет сходные с Apache CouchDB средства для создания документоориентированных баз данных в сочетании с Membase-подобными хранилищами в формате «ключ -- значение». Благодаря поддержке стандартного протокола memcached \cite{memcached}, система остаётся совместимой с большим числом унаследованных приложений и может выступать в роли прозрачной замены ряда других NoSQL-систем. Исходный код системы распространяется под лицензией Apache.

Кроме возможности хранения данных в формате «ключ -- значение», Couchbase позволяет использовать концепцию документоориентированного хранилища, в котором в качестве единицы хранения данных выступает документ, который имеет уникальный идентификатор, версию и содержит произвольный набор именованных полей в формате «ключ -- значение». Используемая модель данных позволяет определять документы в формате JSON, снимая с разработчика необходимость определения схемы хранения. Запросы и индексация данных могут выполняться в соответствии с парадигмой MapReduce. Для организации псевдоструктурированного набора данных из произвольных документов предлагается концепция формирования представлений (view).

\subsubsection{MongoDB}

MongoDB \cite{mongo} -- документоориентированная система управления базами данных, не требующая описания схемы таблиц.

MongoDB реализует подход к построению баз данных без таблиц, схем, SQL-запросов, внешних ключей. В отличие от реляционных баз данных, MongoDB предлагает документо-ориентированную модель данных, благодаря чему MongoDB работает быстрее, обладает лучшей масштабируемостью, ее легче использовать.

Вся система MongoDB может представлять не только одну базу данных, находящуюся на одном физическом сервере. Функциональность MongoDB позволяет расположить несколько баз данных на нескольких физических серверах, и эти базы данных смогут легко обмениваться данными и сохранять целостность.
Способ хранения данных в MongoDB похож на JSON \cite{JSON}. Для хранения в MongoDB применяется формат BSON (binary JSON) \cite{BSON}. BSON позволяет работать с данными быстрее, но данные в JSON-формате занимают меньше места, чем в формате BSON.

\subsection{Выбор СУБД для решения задачи}

Для решения задачи была выбрана СУБД MongoDB, потому что она, на мой взгляд, более проста в эксплуатации, а также поддерживает мультидокументные ACID \cite{ACID} транзакции с изоляцией снапшотов.

\section{Хранение данных о транзакциях}

Каждый пользователь системы может добавить новую выгрузку данных для построения по ней MRR отчета. Хранить данные о выгрузках вместе с пользователем в документоориентированной базе данных не является оптимальным, так как при выгрузке будет получаться ненужная информация о пользователе. Хранить в другой коллекции так же не будет оптимальным, так как будет затруднена агрегация данных.

Для хранения данных о транзакциях необходимо использовать независимую от базы данных пользователей базу данных, причем данная база данных должна быть строго структурированной для унифицированной обработки данных.

\subsection{Классификация баз данных по способу хранения}

По способу хранения базы данных делятся на две группы -- строковые и колоночные базы данных. Каждый из двух типов служит для выполнения определенного рода задач.

\subsubsection{Строковые базы данных (OLTP системы)}

Строковыми базами данных называются базы данных, записи которых в памяти представляются построчно. Строковые базы данных используются в транзакционных системах (OLTP) \cite{oltp}. Для таких систем характерно большое количество коротких транзакций с операциями вставки, обновления и удаления (INSERT, UPDATE, DELETE). Основной упор в системах OLTP делается на очень быструю обработку запросов, поддержание целостности данных в средах с множественным доступом и эффективность, измеряемую количеством транзакций в секунду. В базе данных OLTP есть подробные и текущие данные, а схемой, используемой для хранения транзакционных баз данных, является модель сущностей. Она включает в себя запросы, обращающиеся к отдельным записям, таким как обновление данных клиента в базе данных компании.

\subsubsection{Колоночные базы данных (OLAP системы)}

Колоночными базами данных называются базы данных, записи которых в памяти представляются по столбцам. Колоночные базы данных используются в аналитических системах (OLAP) \cite{olap}. OLAP характеризуется относительно низким объемом транзакций. Запросы часто очень сложны и включают агрегацию. Для систем OLAP время отклика является мерой эффективности. OLAP приложения широко используются методами интеллектуального анализа данных. В базе данных OLAP есть агрегированные, исторические данные, хранящиеся в многомерных схемах. Иногда нужно получить доступ к большому объему данных в управленческих записях, например, какова была прибыль компании в прошлом году.

\subsection{Выбор модели хранения данных для решения задачи}

Для решения задачи колоночное хранение преобладает над строковым по нескольким причинам:

\begin{itemize}
	\item задача не предполагает постоянное добавление и изменение данных. Данные заносятся в базу данных единожды;
	\item задача предполагает выполнение аналитики, следовательно колоночное хранение в приоритете.
\end{itemize}

\subsection{Обзор колоночных СУБД}

Существует большое количество колоночных СУБД. В данном подразделе будут рассмотрены популярные некоммерческие колоночные СУБД, которые могут быть использованы в проекте.

\subsubsection{Greenplum Community Edition}

Greenplum \cite{greenplum} -- это технология больших данных , основанная на архитектуре MPP (massive parallel processing) и PostgresSQL \cite{postgres} с открытым исходным кодом.

Greenplum использует методы массовой параллельной обработки (MPP). Каждый компьютерный кластер состоит из главного узла, резервного главного узла и узлов сегмента. Все данные находятся на узлах сегмента, а информация каталога хранится в главных узлах. Узлы сегментов запускают один или несколько сегментов, которые представляют собой измененные экземпляры базы данных PostgreSQL и которым назначается идентификатор содержимого. Для каждой таблицы данные распределяются между узлами сегмента на основе ключей столбцов распределения, указанных пользователем на языке определения данных. Для каждого идентификатора содержимого сегмента существует как основной, так и зеркальный сегменты, которые не работают на одном физическом хосте. Когда запрос поступает на главный узел, он анализируется, планируется и отправляется всем сегментам для выполнения плана запроса и либо возврата запрошенных данных, либо вставки результата запроса в таблицу базы данных. Язык структурированных запросов используется для представления запросов системе. Семантика транзакции соответствует ACID.

\subsubsection{ClickHouse}

ClickHouse \cite{clickhouse} -- это колоночная аналитическая СУБД с открытым кодом, позволяющая выполнять аналитические запросы в режиме реального времени на структурированных больших данных, разрабатываемая компанией Яндекс.

ClickHouse использует собственный диалект SQL близкий к стандартному, но содержащий различные расширения: массивы и вложенные структуры данных, функции высшего порядка, вероятностные структуры, функции для работы с URI, возможность для работы с внешними key-value хранилищами (<<словарями>>), специализированные агрегатные функции, функциональности для семплирования, приблизительных вычислений, возможность создания хранимых представлений с агрегацией, наполнения таблицы из потока сообщений Apache Kafka и т. д.

Однако при этом имеются и ограничения -- отсутствие транзакций, отсутствие точечных UPDATE/DELETE (пакетный UPDATE/DELETE был введен в июне 2018 года), ограниченная поддержка синтаксиса JOIN, строгие типы с необходимостью явного приведения, для некоторых операций промежуточные данные должны помещаться в оперативную память, отсутствие оконных функций, отсутствие полноценного оптимизатора запросов, точечного чтения, присутствие ограничений в реализации некоторых функций, связанных со спецификой использования ClickHouse в Яндексе, и т. д.

Система оптимизирована для хранения данных на жестких дисках (используются преимущества линейного чтения, сжатия данных). Для обеспечения отказоустойчивости и масштабируемости ClickHouse может быть развернут на кластере (для координации процесса репликации используется Apache ZooKeeper). Для работы с базой данных существует консольный клиент, веб-клиент, HTTP интерфейс, ODBC и JDBC-драйверы, а также готовые библиотеки для интеграции со многими популярными языками программирования и библиотеками.

\subsection{Выбор СУБД для решения задачи}

Для решения задачи была выбрана СУБД ClickHouse, потому что она, имеет различные типы таблиц, которые значительно упростят обработку данных в курсовой работе \cite{chengines}. Помимо этого, в тестах на производительность ClickHouse показал более высокий результат, нежели Greenplum \cite{chperf}.

\section{Хранение данных о вычисленном MRR}

Для того, чтобы повторно не вычислять MRR периода, который уже был вычислен, можно прибегнуть к кэшированию данных. Для кэширования можно прибегнуть к использованию NoSQL in-memory баз данных. Такие базы данных хранят данные в оперативной памяти, что обеспечивает более быстрый доступ, нежели если бы данные хранились на диске.

\subsection{Обзор in-memory NoSQL СУБД}

\subsubsection{Redis}

Redis \cite{redis} -- резидентная система управления базами данных класса NoSQL с открытым исходным кодом, работающая со структурами данных типа «ключ -- значение». Используется как для баз данных, так и для реализации кэшей, брокеров сообщений.

Хранит базу данных в оперативной памяти, снабжена механизмами снимков и журналирования для обеспечения постоянного хранения (на дисках, твердотельных накопителях). Также предоставляет операции для реализации механизма обмена сообщениями в шаблоне «издатель-подписчик»: с его помощью приложения могут создавать каналы, подписываться на них и помещать в каналы сообщения, которые будут получены всеми подписчиками (как IRC-чат). Поддерживает репликацию данных с основных узлов на несколько подчинённых (англ. master -- slave replication). Также поддерживает транзакции и пакетную обработку команд (выполнение пакета команд, получение пакета результатов).

Все данные Redis хранит в виде словаря, в котором ключи связаны со своими значениями. Одно из ключевых отличий Redis от других хранилищ данных заключается в том, что значения этих ключей не ограничиваются строками. Поддерживаются следующие абстрактные типы данных: строки, списки, множества, хеш-таблицы, упорядоченные множества.

Тип данных значения определяет, какие операции (команды) доступны для него; поддерживаются такие высокоуровневые операции, как объединение и разность наборов, сортировка наборов.

\subsubsection{Tarantool}

Tarantool \cite{tarantool} -- это платформа in-memory вычислений с гибкой схемой данных для эффективного создания высоконагруженных приложений. Включает в себя базу данных и сервер приложений на Lua.

Обладает высокой скоростью работы по сравнению с традиционными СУБД, обладая теми же свойствами: персистентности, транзакционности ACID, репликации master-slave, master-master.

Для хранения данных используются таплы (кортежи). Это массив с данными, которые не типизированы. Кортежи или таплы объединяются в спейсы. Спейс – это аналог из мира SQL, таблица. Спейс это коллекция таплов, а тапл это коллекция полей.

Кортежи организованы в пространства (space или таблицы). Для каждого пространства указывается технология хранения (memtx или vinyl).

Пространство должно быть проиндексировано первичным ключом. Также поддерживаются неограниченное количество вторичных ключей.

Ключ может состоять из одного и более полей.

\subsection{Выбор СУБД для решения задачи}

Для кэширования данных выбрана СУБД Tarantool, потому как она проста в развертывании, а также имеет исчерпывающую документацию на русском языке.

\section{Формализация данных}

В данном подразделе описаны требования к данным, содержащимся в базах данных, используемых при разработке приложения.

\subsection{База данных пользователей}

База данных пользователей должна хранить информацию о пользователях системы. Каждый пользователь должен обладать уникальным идентификатором, чтоы можно было однозначно идентифицировать пользователя.

\subsection{База данных транзакций}

База данных транзакций должна хранить информацию о каждой проведенной транзакции, причем неважно однозначно идентифицировать транзакцию, потому как они будут выбираться в промежутке между какими-то датами. Для того, чтобы использовать одну таблицу для транзакций всех пользователей, транзакцию можно помечать специальным признаком, который позволит в выборке брать транзакции, принадлежащие только текущему пользователю.

Помимо этого база должна хранить промежуточные таблицы распределения уплаченных средств по периодам.

\subsection{База данных кэшируемого результата}

База данных кэшируемого результата должна хранить конечный результат для прямого визуализирования, без дополнительной обработки.

\section{Анализ существующих решений}

В качестве существующих решений для анализа выбраны сервисы Pabbly Subscriptions Billing \cite{pbsubbil}, SaaS Metrics \cite{saas}, ChartMogul \cite{chartmogul}, Databox \cite{databox} и Baremetrics \cite{baremetrics}.

В таблице \ref{tab:solutions} представлен сравнительный анализ существующих решений.

\begin{table}[H]
	\centering
	\caption{Анализ существующих решений}
	\label{tab:solutions}
	\resizebox{\textwidth}{!}{%
		\begin{tabular}{|c|c|c|c|c|}
			\hline
			\textbf{Сервис}                                                            & \textbf{Цена}                                                         & \textbf{\begin{tabular}[c]{@{}c@{}}Аналитика \\ продаж\end{tabular}} & \textbf{\begin{tabular}[c]{@{}c@{}}Аналитика \\ подписок\end{tabular}} & \textbf{\begin{tabular}[c]{@{}c@{}}Безлимитный \\ расчет MRR\end{tabular}} \\ \hline
			\begin{tabular}[c]{@{}c@{}}Pabbly \\ Subscriptions \\ Billing\end{tabular} & От \$19 в месяц                                                       & Да                                                                   & Да                                                                     & Да                                                                         \\ \hline
			SaaS Metrics                                                               & \begin{tabular}[c]{@{}c@{}}Свяжитесь с \\ отделом продаж\end{tabular} & Нет                                                                  & Нет                                                                    & Нет                                                                        \\ \hline
			ChartMogul                                                                 & От \$125 в месяц                                                      & Да                                                                   & Да                                                                     & Нет                                                                        \\ \hline
			Databox                                                                    & От \$59 в месяц                                                       & Да                                                                   & Да                                                                     & Нет                                                                        \\ \hline
			Baremetrics                                                                & От \$50 в месяц                                                       & Да                                                                   & Да                                                                     & Нет                                                                        \\ \hline
		\end{tabular}%
	}
\end{table}

\section*{Вывод}

В данном разделе была рассмотрена метрика MRR, задача построения MRR отчета. Были проанализированы способы хранения информации для компонентов системы, а также выбраны оптимальные способы для решения потавленной задачи. Был проведен анализ СУБД, используемых для решения задачи и также выбраны оптимальные информационные системы. Также были формализованы данные, используемые в системе.