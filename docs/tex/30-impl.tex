\chapter{Технологическая часть}

В данном разделе представлены архитектура приложения, средства разработки программного обеспечения, детали реализации и пользовательский интерфейс.

\section{Архитектура приложения}

Пользователь взаимодействует с системой посредством Web-приложения, построенного с использованием паттерна SPA (Single Page Application) \cite{spa}.

Пользовательская часть приложения коммуницирует с серверной частью при помощи реализованного RPC API на стороне сервера \cite{rpc} \cite{API}.

Серверная часть коммуницирует с базами данных при помощи специализированных коннекторов, позволяющих делать запросы к базе данных из языка программирования.

Вся обработка данных производится на сервере, чтобы не загружать клиентское приложение и не вызывать задержек в обработке.

Общая схема архитектуры приложения представлена на рисунке \ref{img:cp_arch}.

\img{70mm}{cp_arch}{Схема архитектуры приложения}

\section{Средства реализации}

Для разработки пользовательского интерфейся был выбран JavaScript фреймворк \texttt{Vue.js 3} \cite{js} \cite{vue}. Vue.js позволяет создавать отзывчивые SPA приложения, основанные на компонентах. Компоненты впоследствии можно переиспользовать без дублирования кода. Помимо этого, при помощи специальных библиотек: \texttt{vuex} и \texttt{vue-router} можно оптимально управлять данными внутри приложения и имитировать многостраничное приложение соответственно \cite{vuex} \cite{vuerouter}.

Для разработки серверной части приложения был выбран язык \texttt{Go} \cite{golang}. Выбор обусловлен тем, что данный язык является компилируемым, следовательно в процессе разработке возникнет меньше промежуточных ошибок. Также Go предоставляет нативную поддержку многопоточности, что позволяет повысить производительность приложения. В качестве фреймворка для реализации API был выбран \texttt{gin} \cite{gin}. gin предоставляет наибольшую производительность среди существующих Go фреймворков для реализации REST/RPC API \cite{ginbench}.

Для коммуницирования серверной части приложения с базами данных были использованы официальные коннекторы: \texttt{go-tarantool} для Tarantool, \texttt{dbr} для ClickHouse и \texttt{mongo-go-driver} для MongoDB \cite{gotnt} \cite{goch} \cite{gomongo}.

Для упаковки приложения в готовый продукт была выбрана система контейнеризации \texttt{Docker} \cite{docker}. Docker позволяет создать изолированную среду для программного обеспечения, которое можно будет разворачивать на различных операционных системах без дополнительно вмешательства для обеспечения совместимости.

\section{Детали реализации}

В листингах \ref{lst:clientserver} -- \ref{lst:cachedb} представлены листинги реализации взаимодействия клиента с сервером, обработка запроса с клиента на сервере, подключение и инициализация базы данных пользователей, транзакций, кэшированных данных соответственно.

\begin{lstinputlisting}[
	caption={Пример реализации взаимодействия клиента с сервером},
	label={lst:clientserver},
	style={ES6},
	linerange={2-18},
	]{/Users/s.kononenko/Study/mrratality/frontend/src/store/modules/analytics/actions.js}
\end{lstinputlisting}

\begin{lstinputlisting}[
	caption={Пример реализации обработки запроса клиента на сервере},
	label={lst:serverclient},
	style={go},
	linerange={72-153},
	]{/Users/s.kononenko/Study/mrratality/backend/server/controllers/files.go}
\end{lstinputlisting}

\begin{lstinputlisting}[
	caption={Инициализация базы данных пользователей},
	label={lst:userdb},
	style={go},
	linerange={26-47},
	]{/Users/s.kononenko/Study/mrratality/backend/db/user/utils.go}
\end{lstinputlisting}

\begin{lstinputlisting}[
	caption={Инициализация базы данных транзакций},
	label={lst:storagedb},
	style={go},
	linerange={16-34,67-94},
	]{/Users/s.kononenko/Study/mrratality/backend/db/storage/utils.go}
\end{lstinputlisting}

\begin{lstinputlisting}[
	caption={Инициализация базы данных кэширования},
	label={lst:cachedb},
	style={go},
	linerange={13-56},
	]{/Users/s.kononenko/Study/mrratality/backend/db/cache/utils.go}
\end{lstinputlisting}

\section{Пользовательский интерфейс}

На рисунке \ref{img:cp_mainpage} представлена главная страница Web-приложения. Главную страницу может просматривать любой пользователь.

\img{90mm}{cp_mainpage}{Главная страница приложения}

На рисунке \ref{img:cp_login} представлена страница входа. Страницу входа может просматривать любой неавторизованный пользователь. При попытке авторизованного пользователя посетить страницу входа, пользователь перенаправляется на главную страницу.

\img{90mm}{cp_login}{Страница входа приложения}

На рисунках \ref{img:cp_loginfailed} -- \ref{img:cp_loginpass} представлен контроль над корректностью даных для входа в приложение. Если в базе данных пользователей не найден пользователь с соответствующим адресом электронной почты, то будет выдано соответствующее сообщение. Аналогично в случае, если пользователь указал неверный пароль.

\img{90mm}{cp_loginfailed}{Ошибка входа из-за несуществующего адреса электронной почты}

\img{90mm}{cp_loginpass}{Ошибка входа из-за неверного пароля}

На рисунке \ref{img:cp_loggedin} представлен вид личного кабинета пользователя, в котором он может загружать новые файлы, удалять ненужные и выбирать файлы для аналитики.

\img{90mm}{cp_loggedin}{Личный кабинет пользователя}

На рисунке \ref{img:cp_files} представлен вид личного кабинета пользователя, который загрузил несколько файлов в систему. Пользователь может ориентироваться в файлах по дате загрузки.

\img{90mm}{cp_files}{Личный кабинет пользователя с загрженными файлами}

На рисунке \ref{img:cp_analytics} представлена страница для выбора периода для аналитики выбранного файла.

\img{90mm}{cp_analytics}{Выбор периода для аналитики}

На рисунках \ref{img:cp_control} -- \ref{img:cp_nodata} представлен контроль ошибок на данном этапе: пользователь оповещается об ошибках в случаях, если пользователь выбрал окончание периода, дата которого раньше начала периода, и если в выбранный период не было данных.

\img{90mm}{cp_control}{Ошибка аналитики из-за неправильных границ периода}

\img{90mm}{cp_nodata}{Ошибка аналитики из-за отсутствия данных в выбранный период}

На рисунках \ref{img:cp_chart} -- \ref{img:cp_table} представлены результаты построения MRR отчета в виде гистограммы и в виде таблицы. Гистограмма является интерактивной: при наведении курсора мыши можно увидеть дополнительные сведения.

\img{90mm}{cp_chart}{MRR отчет в виде гистограммы}

\img{90mm}{cp_table}{MRR отчет в виде таблицы}

\section*{Вывод}

В данном разделе были представлены средства реализации программного обеспечения, листинги ключевых компонентов системы а также представлен пользовательский интерфейс приложения.
