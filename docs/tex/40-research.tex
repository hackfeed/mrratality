\chapter{Исследовательская часть}

В данном разделе представлена постановка эксперимента по сравнению занимаемого времени для получения итоговых данных с и без использования кэширования.

\section{Постановка эксперимента}

В данном подразделе представлены цель, описание и результаты эксперимента.

\subsection{Цель эксперимента}

Целью эксперимента является сравнение времени, требуемого для получения итоговых данных для построения MRR отчета с и без использования кэширования.

\subsection{Описание эксперимента}

Сравнить занимаемое время можно при помощи отключения реализованного механизма кэширования.

Для этого нужно отключить базу данных, хранящую данные о кэшировании, и каждый раз выполнять запрос напрямую к базе данных с транзакциями.

Для проведения эксперимента будут использоваться данные, по которым можно построить годовой MRR отчет (12 месяцев). Тестирование будет проводиться на периодах 36 месяцев, 24 месяца, 12 месяцев, 6 месяцев, 3 месяца, 1 месяц.

\subsection{Результат эксперимента}

В таблице \ref{tab:experiment} представлены результаты поставленного эксперимента.

\begin{table}[H]
	\centering
	\caption{Результаты сравнения времени, необходимого для получениях данных без кэширования и с кэшированием}
	\label{tab:experiment}
	\resizebox{\textwidth}{!}{%
		\begin{tabular}{|c|c|c|}
			\hline
			\textbf{\begin{tabular}[c]{@{}c@{}}Количество \\ месяцев\end{tabular}} & \textbf{Время без кэширования, мс} & \textbf{Время с кэшированием, мс} \\ \hline
			1                                                                      & 576.69                             & 3.65                              \\ \hline
			3                                                                      & 736.40                             & 2.85                              \\ \hline
			6                                                                      & 647.37                             & 2.48                              \\ \hline
			12                                                                     & 647.01                             & 3.08                              \\ \hline
			24                                                                     & 628.51                             & 2.54                              \\ \hline
			36                                                                     & 661.81                             & 2.90                              \\ \hline
		\end{tabular}%
	}
\end{table}

\section*{Вывод}

В результате сравнения времени, необходимого для получения данных для построения MRR отчета, алгоритм с кэшированием выигрывает у алгоритма без кэширование примерно в 250 раз. Данный результат достигается за счет того, что в приложении база данных с кэшем хранит уже готовые для отчета данные, не требующие дополнительнойо бработки, в то время как база данных транзакций в конечном выиде содержит лишь нормированное распределение уплаченных клиентом средств. Несмотря на тот факт, что в приложении в СУБД ClickHouse используется движок Memory, хранящий данные в оперативной памяти, алгоритм с кэшированием все равно выигрывает за счет асимптотической сложности поиска по индексу $O(1)$, что означает, что при любом объеме выборки будет получен одинаковый результат затрачиваемого времени.
