\chapter{Исследовательская часть}

В данном разделе представлены примеры работы программного обеспечения и постановка эксперимента по сравнению результата изменения цветовой палитры изображения методом наивной инверсии и методом изменения компонентов на основе анализа цветовой карты.

\section{Пример работы программного обеспечения}

На рисунках \ref{img:page1} -- \ref{img:page3ch} приведены исходные состояния веб-страниц и состояния веб-страниц с измененной цветовой палитрой соответственно.

Также на рисунках \ref{img:bmstu} и \ref{img:bmstuch} приведены исходное и измененное состояния официального сайта МГТУ им. Н. Э. Баумана.

\clearpage

\img{100mm}{page1}{Исходное состояние веб-страницы 1}

\img{100mm}{page1ch}{Cостояние веб-страницы 1 с измененной цветовой палитрой}

\img{100mm}{page2}{Исходное состояние веб-страницы 2}

\img{100mm}{page2ch}{Cостояние веб-страницы 2 с измененной цветовой палитрой}

\img{100mm}{page3}{Исходное состояние веб-страницы 3}

\img{100mm}{page3ch}{Cостояние веб-страницы 3 с измененной цветовой палитрой}

\img{100mm}{bmstu}{Исходное состояние веб-страницы МГТУ}

\img{100mm}{bmstuch}{Cостояние веб-страницы МГТУ с измененной цветовой палитрой}

На рисунках \ref{img:load1} -- \ref{img:load3} приведены данные загрузки веб-страниц. Разработанное программное обеспечение, как и планировалось, производит преобразование в течение времени до двух секунд, что удовлетворяет поставленному условию.

\boximg{60mm}{load1}{Затраченные ресурсы при загрузке страницы 1}

\boximg{60mm}{load2}{Затраченные ресурсы при загрузке страницы 2}

\boximg{60mm}{load3}{Затраченные ресурсы при загрузке страницы 3}

\section{Постановка эксперимента}

В данном подразделе представлены цель, описание и результаты эксперимента.

\subsection{Цель эксперимента}

Целью эксперимента является сравнение методов изменения цветовой палитры изображения. Критерием сравнения будет являться контрастность полученного изображения \cite{wcag1}.

\subsection{Описание эксперимента}

Сравнить результат преобразования цветовой палитры изображения можно при помощи сравнения полученной контрастности \cite{wcagcontrast}.

Хорошим считается отношение 4.5:1 и больше, отличным -- 7:1 и больше для самого яркого и самого темного цветов страницы. Так же стоит учитывать контрастность смежных элементов, которая должна быть не меньше 3:1.

\subsection{Результат эксперимента}

В таблице \ref{tbl:experiment} представлены результаты поставленного эксперимента. СЯ-СТ -- контрастность для самого яркого и самого темного цветов, СМЕЖ -- минимальная вычисленная контрастность смежных элементов.

\begin{table}[H]
	\caption{Результаты сравнения методов изменения цветовой палитры изображения}
	\centering
	\resizebox{\textwidth}{!}{%
		\label{tbl:experiment}
		\begin{tabular}{|l|l|l|l|l|}
			\hline
			\multicolumn{1}{|c|}{\multirow{2}{*}{\textbf{Веб-страница}}} &
			\multicolumn{2}{c|}{\textbf{Наивная инверсия}} &
			\multicolumn{2}{c|}{\textbf{Анализ цветовой карты}} \\ \cline{2-5} 
			\multicolumn{1}{|c|}{} &
			\multicolumn{1}{c|}{\textbf{СЯ-СТ}} &
			\multicolumn{1}{c|}{\textbf{СМЕЖ}} &
			\multicolumn{1}{c|}{\textbf{СЯ-СТ}} &
			\multicolumn{1}{c|}{\textbf{СМЕЖ}} \\ \hline
			WineChecker & 21:1   & 21:1   & 13.92:1 & 13.92:1 \\ \hline
			InK         & 21:1   & 3.74:1 & 13.92:1 & 3.26:1  \\ \hline
			BLOGger     & 1.61:1 & 1.34:1 & 13:92:1 & 3.26:1  \\ \hline
		\end{tabular}%
	}
\end{table}

\section*{Вывод}

В результате сравнения методов изменения цветовой палитры изображения было выявлено, что метод наивной инверсии может давать требуемую контрастность, не во всех случаях. Метод изменения цветовой палитры на основе анализа цветовой карты позволяет добиться нужной контрастности в любом случае, потому что метод предполагает собой сравнение и анализ полученных цветов изображения и выбор наиболее подходящих цветов для изменения.
