\chapter*{Введение}
\addcontentsline{toc}{chapter}{Введение}

Бизнес-модель подписки -- это бизнес-модель, в которой клиент должен регулярно (с определенной периодичностью) платить фиксированную сумму, чтобы получить доступ к продукту \cite{subscriptionbm}.

По состоянию на 2021 год 55\% людей в мире имеют подписку на услуги более чем одного сервиса из <<большой тройки>> (Netflix, Amazon Prime, Hulu)  \cite{subscriptioncons}. И это только сервисы видеостриминга. Около 400 млн. людей имеют подписку на сервисы аудиостриминга \cite{audiostreaming}.

Пандемия COVID-19 в 2020 году положительно повлияла на развитие бизнес-модели подписки: количество поисковых запросов о подписках на видеостриминговые сервисы в среднем выросло на 102\% \cite{videostreamingsearch}, а количество пользователей, приобретших подписки на хотя бы один такой сервис, выросло в среднем на 20\% \cite{subgrowth}.

Несмотря на то, что пандемия отступает, бизнес-модель подписки только развивает свою популярность, приобретенную в период пандемии. По прогнозам, потраченные пользователями в 2021 году 4054 млн. долларов на подписки вырастут до 7813 млн. долларов в 2025 году \cite{subforecast}.

Чтобы иметь полный контроль над денежными потоками в приложениях с моделью подписок, бизнесу нужны соответствующие инструменты, чтобы анализировать, планировать и прогнозировать движения средств. 

Учесть течение финансов можно при помощи метрики MRR (Monthly Recuring Revenue) \cite{mrr}. Существуют инструменты, которые позволяют вычислять значение данной метрики для заданного набора данных. Основной их недостаток в том, что они коммерческие и доступны не каждому бизнесу, желающему провести анализ и последующий прогноз финансовых потоков.

Цель работы -- реализовать программное обеспечение для определения MRR приложения по заданному набору данных. 

Чтобы достигнуть поставленной цели, требуется решить следующие задачи:
\begin{itemize}
	\item определить методы вычисления MRR;
	\item проанализировать варианты представления данных для аналитики и выбрать подходящий вариант для решения задачи;
	\item проанализировать системы управления базами данных и выбрать подходящую систему для хранения данных;
	\item спроектировать базу данных, описать ее сущности, связи;
	\item реализовать интерфейс для доступа к базе данных;
	\item реализовать программное обеспечение, которое предоставит пользователю доступ к построенным отчетам в графической и табличной форме.
\end{itemize}